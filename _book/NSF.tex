\documentclass[]{book}
\usepackage{lmodern}
\usepackage{amssymb,amsmath}
\usepackage{ifxetex,ifluatex}
\usepackage{fixltx2e} % provides \textsubscript
\ifnum 0\ifxetex 1\fi\ifluatex 1\fi=0 % if pdftex
  \usepackage[T1]{fontenc}
  \usepackage[utf8]{inputenc}
\else % if luatex or xelatex
  \ifxetex
    \usepackage{mathspec}
  \else
    \usepackage{fontspec}
  \fi
  \defaultfontfeatures{Ligatures=TeX,Scale=MatchLowercase}
\fi
% use upquote if available, for straight quotes in verbatim environments
\IfFileExists{upquote.sty}{\usepackage{upquote}}{}
% use microtype if available
\IfFileExists{microtype.sty}{%
\usepackage{microtype}
\UseMicrotypeSet[protrusion]{basicmath} % disable protrusion for tt fonts
}{}
\usepackage[margin=1in]{geometry}
\usepackage{hyperref}
\hypersetup{unicode=true,
            pdftitle={Arsip The Werewolf Game Ninja Saga Forum SF Indonesia},
            pdfauthor={Tmofer},
            pdfborder={0 0 0},
            breaklinks=true}
\urlstyle{same}  % don't use monospace font for urls
\usepackage{natbib}
\bibliographystyle{plainnat}
\usepackage{longtable,booktabs}
\usepackage{graphicx,grffile}
\makeatletter
\def\maxwidth{\ifdim\Gin@nat@width>\linewidth\linewidth\else\Gin@nat@width\fi}
\def\maxheight{\ifdim\Gin@nat@height>\textheight\textheight\else\Gin@nat@height\fi}
\makeatother
% Scale images if necessary, so that they will not overflow the page
% margins by default, and it is still possible to overwrite the defaults
% using explicit options in \includegraphics[width, height, ...]{}
\setkeys{Gin}{width=\maxwidth,height=\maxheight,keepaspectratio}
\IfFileExists{parskip.sty}{%
\usepackage{parskip}
}{% else
\setlength{\parindent}{0pt}
\setlength{\parskip}{6pt plus 2pt minus 1pt}
}
\setlength{\emergencystretch}{3em}  % prevent overfull lines
\providecommand{\tightlist}{%
  \setlength{\itemsep}{0pt}\setlength{\parskip}{0pt}}
\setcounter{secnumdepth}{5}
% Redefines (sub)paragraphs to behave more like sections
\ifx\paragraph\undefined\else
\let\oldparagraph\paragraph
\renewcommand{\paragraph}[1]{\oldparagraph{#1}\mbox{}}
\fi
\ifx\subparagraph\undefined\else
\let\oldsubparagraph\subparagraph
\renewcommand{\subparagraph}[1]{\oldsubparagraph{#1}\mbox{}}
\fi

%%% Use protect on footnotes to avoid problems with footnotes in titles
\let\rmarkdownfootnote\footnote%
\def\footnote{\protect\rmarkdownfootnote}

%%% Change title format to be more compact
\usepackage{titling}

% Create subtitle command for use in maketitle
\providecommand{\subtitle}[1]{
  \posttitle{
    \begin{center}\large#1\end{center}
    }
}

\setlength{\droptitle}{-2em}

  \title{Arsip The Werewolf Game Ninja Saga Forum SF Indonesia}
    \pretitle{\vspace{\droptitle}\centering\huge}
  \posttitle{\par}
    \author{Tmofer}
    \preauthor{\centering\large\emph}
  \postauthor{\par}
      \predate{\centering\large\emph}
  \postdate{\par}
    \date{2019-05-09}

\usepackage{booktabs}

\begin{document}
\maketitle

{
\setcounter{tocdepth}{1}
\tableofcontents
}
\chapter*{Kata Pengantar}\label{kata-pengantar}
\addcontentsline{toc}{chapter}{Kata Pengantar}

\chapter{Turnabout Werewolves}\label{turnabout-werewolves}

\section{Prolog}\label{prolog}

``\emph{We might be the werewolves ..but it doesn't necessarily mean we
are always the bad guys. Not this time.}''

\begin{figure}
\centering
\includegraphics{Gambar/01-cover}
\caption{Cover}
\end{figure}

Dunia telah mencapai awal abad 21.

Jujur saja, kami ragu jika masyarakat kota masih berpikir bahwa kami
masih ada di sini.. Sengaja hidup dalam keterasingan walau sebetulnya
jarak antara hutan ini tak terlampau jauh dari peradaban.. Tapi, hei,
buat apa kami mengganggu mereka?

Kalau boleh jujur sekali lagi, kami sudah lelah. Dan bosan. Lelah dan
bosan menyantap daging mereka.

\emph{Oh, tambahkan lagi satu kata, jijik.}

Mereka hanyalah onggokan daging yang kotor yang berjalan, walau bukan
dalam arti harafiah.. karena kami pikir, apa yang mereka sebetulnya
lakukan itu jauh lebih licik dan hina daripada kami.

Meraup untung dan mencuri setiap keping dan lembar apa yang bukan milik
mereka. Tertawa bahagia atas kefanaan harta benda yang mereka miliki
saat ini. Tak jarang pula mereka mencekoki para penduduk desa yang
kurang berpendidikan dengan cerita bohong tentang kami--bagaimana kami
mencoba hidup di tengah mereka dengan bermuka-dua pada siang hari, dan
lantas menyantap mereka hidup-hidup pada malam harinya.

Sekadar membuat orang-orang untuk takut pada apa yang nyata terlihat,
namun tidak awas pada apa yang tak nampak di luarnya.

Yeah.

Oleh karena itulah kami sudah berhenti membunuhi manusia sejak berpuluh
tahun yang lalu. Kami memutuskan untuk tinggal jauh dari peradaban,
mencoba untuk membiarkan mereka hidup dengan apa yang mereka sebut
sebagai ``akan lebih damai jika tidak ada makhluk seperti kalian di
sini''. Oke, biar kami lihat dunia seperti apa yang kalian, manusia,
ingin wujudkan. Dunia yang pada kenyataannya begitu kotor, yang bahkan
kami pun muak hanya sekadar untuk membayangkannya.

..Tapi lucunya, walaupun kami telah menghentikan perlawanan kami
terhadap manusia, kadang kami masih mendengar para penduduk kota
membisikkan kabar berita, bahwa selalu ada korban manusia serigala
setiap bulan purnama, penuh dengan bekas luka cabikan dan sayatan benda
tajam--padahal setahu kami, tidak ada satu pun dari kami yang melakukan
hal tersebut. Dan anehnya lagi, setiap ada satu korban manusia jatuh
pada bulan purnama, satu dari kami pun akan ditemukan mati di dekat
hutan dengan luka tembak di bagian dada.

Segala rangkaian peristiwa ini seolah menantang kami dan mengatakan,
``\emph{Kenapa berhenti membunuh manusia, eh? Dasar pengecut. Bukankah
ini adalah yang dulu biasa kalian lakukan?}''

Sejak saat itu jumlah kami semakin sedikit pada setiap bulan purnama,
dan kini kami tak lagi bisa tinggal diam hanya menunggu ajal misterius
itu untuk menjemput kami. Kami memutuskan agar salah seorang dari kami,
\textbf{cobaltblue}, untuk menginvestigasi hal ini lebih lanjut.
\textbf{cobaltblue} menyusup ke tengah kota dan menyamar sebagai salah
satu karyawan di suatu perusahaan yang cukup terkenal dengan
perlawanannya pada kaum kami di masa lampau untuk mencari informasi. Dan
untunglah, usaha tersebut tidak sia-sia. Kami ingat dulu dulu pernah ada
satu dari kelompok kami yang hilang tanpa jejak. Seorang manusia
serigala yang masih terhitung muda, dan ternyata kawan kecil kami inilah
biang keladi dari segala peristiwa ini. Dia telah beralih menjadi agen
rahasia pembasmi manusia serigala dan bekerja sebagai pengawal pribadi
direktur perusahaan Gninnuc.

Dia yang dulu pernah menjadi bagian dari kami. Dia yang jelas mengetahui
identitas kami.

Dia yang berkhianat dua kali. Pertama, ketika dia masih bagian dari kami
dan \emph{berkhianat} pada manusia. Kedua, ketika dia beralih haluan
ingin menghabisi kami. Berkhianat pada kami untuk pihak kotor perusahaan
Gninnuc, bah. Perusahaan yang mengelola jasa asuransi dan layanan
masyarakat, katanya.. Padahal kami tahu, perusahaan itu sebetulnya
memproduksi dan menyeludupkan narkotika.

Satu lagi sampah busuk dan kotor dari pihak manusia. Ah.. ya. Mungkin
ini kesempatan kami, untuk sekadar \emph{mengabdikan} diri pada prinsip
kami yang baru.

Kami akan membunuh pengkhianat licik itu, sekaligus menghabisi direktur
konyol beserta dengan para anak buahnya.

Maka kami pun sepakat, ini akan menjadi langkah awal dari sebuah
revolusi kami.

Namun ada satu hal kecil yang membuat semua ini menjadi sulit. Direktur
ini adalah seorang indigo, yang memiliki indera ke-enam sehingga dia
mampu mengetahui identitas kami bahkan ketika kami mencoba menyamar
sebagai penduduk kota biasa. Dan benar saja, ketika kami baru saja
menerima informasi berharga mengenai kawan pengkhianat kami kemarin,
mendadak pada pagi harinya, kami telah menemukan \textbf{cobaltblue}
tewas, lagi-lagi dengan luka tembak; dan beberapa luka sayatan dan
gigitan--yang berarti sangat mungkin bahwa pengkhianat kecil ini sempat
melakukan serangan ketika dia sedang bertransformasi.

Ini tidak bisa dibiarkan lagi.

Mungkin kami kalah jumlah. Tapi mereka akan butuh lebih dari sekadar
jumlah untuk melenyapkan kami.

Mungkin kami memang manusia serigala. Tapi kalian akan butuh lebih dari
sekadar bulu dan taring untuk membuktikan bahwa kami adalah penjahatnya.

\emph{Yah. Setidaknya, untuk saat ini.}

\section{Setup}\label{setup}

\subsection{Peran}\label{peran}

\begin{itemize}
\item
  (Villager) - Karyawan perusahaan: Pemain biasa, tanpa kemampuan
  khusus. Tugas mereka adalah hanya melakukan vote pada siang hari.
\item
  (Masons) - Sekretaris dan Asisten Direktur: Pemain biasa, tanpa
  kemampuan khusus. Tugas mereka sama seperti Villager, tapi yang
  berbeda adalah, tiap Mason akan tahu identitas Mason yang lainnya;
  sehingga bisa dikatakan, Mason bisa bekerja dalam tim.
\item
  (Werewolves) - Lone Werewolves: Nantinya juga akan melakukan voting
  pada siang hari. Kemampuan khusus bagi peran manusia serigala adalah
  kesempatan untuk membunuh satu orang untuk dibunuh pada malam hari.
  Para manusia serigala nantinya akan berdiskusi mengenai siapa yang
  ingin mereka bunuh, kemudian mengirimkan nama calon korban pada GM
  melalui PM. Para manusia serigala juga tahu identitas manusia serigala
  lainnya.
\item
  (Seer) - Direktur Gninnuc.Corp: Sama, nantinya juga akan melakukan
  voting pada siang hari. Kemampuan khusus pada malam hari bagi peran
  Seer adalah kesempatan untuk meminta GM melalui PM untuk `melihat'
  peran dari seorang player.
\item
  (Vigilante) - Pengawal Pribadi Direktur: *Peran ini akan dihilangkan
  dari game sejak round pertama; sebagai gantinya akan ada item -
  ``Silver Riffle''. Pada malam hari, seorang Vigilante diperbolehkan
  untuk membunuh SATU nama saja yang dia curigai sebagai manusia
  serigala. Namun karena dia hanya punya satu peluru, itu berarti dia
  hanya punya kesempatan untuk melakukan 'kemampuan khusus'nya sekali
  saja.
\end{itemize}

\subsection{Item: Silver Riffle}\label{item-silver-riffle}

Item khusus ini akan muncul di tengah game sebagai barang peninggalan
Vigilante yang telah mati sejak awal game. Item ini harus dioper pada
player lain tiap kali malam berakhir. Dalam Silver Riffle ini hanya ada
satu peluru perak sehingga walaupun item ini mungkin akan selalu
berpindah tangan, penggunaannya sendiri hanya bisa sekali saja.

\begin{itemize}
\item
  Silver Riffle hanya bisa digunakan pada malam hari.
\item
  Silver Riffle tidak boleh disimpan lebih dari satu hari oleh satu
  pemain yang sama.
\item
  Silver Riffle baru boleh dipegang oleh orang yang sama setelah lewat
  dua hari. Jadi jika Silver Riffle dipegang oleh pemain A, item tsb
  harus dioper kepada minimal dua pemain terlebih dahulu sebelum pemain
  A bisa memegangnya kembali.
\item
  Perpindahan Silver Riffle akan dilakukan player tanpa perantara GM
  melalui forum - yang berarti semua player nantinya akan mengetahui
  siapa pemegang Silver Riffle yang sebelumnya.
\item
  Jika Silver Riffle jatuh ke tangan werewolves, werewolves akan tetap
  menyimpannya tapi mereka tidak bisa menggunakannya (jadi dua kematian
  anggota tim villager dalam satu malam tidak akan mungkin terjadi).
\item
  Jika Silver Riffle dipegang oleh anggota tim Villager, maka item juga
  akan berfungsi sebagai ``pelindung'' pada malam hari. Jika werewolf
  kebetulan menyerang seorang pemegang Silver Riffle pada malam harinya,
  maka peluru perak akan secara otomatis digunakan untuk membunuh sang
  werewolf - sehingga memungkinkan untuk menghasilkan ending `Seri' atau
  `Draw'.
\item
  Silver Riffle akan musnah begitu menembakkan satu peluru perak.
\item
  Khusus pada kasus anggota tim villager sebagai pemegang Silver Riffle,
  ketika malam itu dia diterkam, Silver Riffle memang secara otomatis
  membunuh werewolf sehingga dia selamat pada malam itu, namun akan mati
  pada esok harinya. Jadi mekanisme perlindungan ini lebih seperti
  `menunda kematian'.
\end{itemize}

\textbf{Mekanisme Perpindahan Silver Riffle:}

\begin{itemize}
\item
  Silver Riffle pertama kali akan diberikan oleh GM pada seorang player
  secara random melalui PM.
\item
  Silver Riffle ini boleh dipindahtangankan setiap kali sesi Siang Hari
  dimulai. Konfirmasi yang dilakukan setelah deadline tidak akan
  dihitung.
\item
  Contoh pesan berpindahtangannya Silver Riffle melalui forum:
  \textbf{Silver Riffle diberikan kepada Mikazuki Gryphin.}
\end{itemize}


\end{document}
